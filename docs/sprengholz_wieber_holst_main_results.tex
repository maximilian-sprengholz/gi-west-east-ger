\documentclass[a4paper,11pt]{scrartcl}
\usepackage[none]{hyphenat}
\usepackage{natbib}
\usepackage[latin1]{inputenc}
\usepackage[english]{babel}
\usepackage[T1]{fontenc}
\usepackage{lmodern}
\usepackage{amsmath}
\providecommand{\twocite}[4]{\citep[{\citealp[#1]{#2};}][#3]{#4}}
\usepackage{romannum}
\usepackage{amsmath}
\usepackage{graphicx}
% captions
\usepackage{caption}
\captionsetup[table]{
	labelsep=period,
	justification=raggedright,
	format=plain,
	textfont={footnotesize, it},
	name=Table,
	labelfont={footnotesize},
	skip=3pt,
	belowskip=0pt,
	position=top
}
\captionsetup[figure]{
	labelsep=period,
	justification=raggedright,
	format=plain,
	textfont={scriptsize, sc},
	name=Figure,
	labelfont={scriptsize, sc},
	skip=15pt,
	belowskip=0pt,
	position=bottom,
	width=13cm
}
%\renewcommand{\thetable}{\Roman{table}}
%\renewcommand{\thefigure}{\Roman{figure}}
\setlength{\parindent}{0em}
\usepackage{ngerman}
\usepackage{ dsfont }
\usepackage{color}
\definecolor{gray}{gray}{0.60}
\usepackage[paper=a4paper,left=25mm,right=25mm,top=25mm,bottom=25mm]{geometry}
\usepackage{fancyhdr}
\usepackage[nolist]{acronym}%Abkürzungen
\usepackage{pdflscape}
\usepackage{tcolorbox} % für boxen um hypothesen
\usepackage{afterpage}
\usepackage{capt-of}
\usepackage{enumitem}
\usepackage{longtable}
\usepackage{ marvosym }
\usepackage{booktabs} % tabout
\usepackage{tabularx} % tabout
\usepackage{subfig} % Mehrere Abbildungen nebeneinander
\usepackage[outdir=./]{epstopdf}
\tcbuselibrary{theorems}
\usepackage{csquotes}
\usepackage{float}
\usepackage{setspace}
% toc
\usepackage[toc,page]{appendix}
% tabout/tables
\usepackage{booktabs} % tabout
\usepackage{tabularx} % tabout
\usepackage{siunitx} % table cell content alignment decimal point
\def\sym#1{\ifmmode^{#1}\else\(^{#1}\)\fi} % Signifikanz-Sternchen
\usepackage[usestackEOL]{stackengine} % table cell with line breaks
\usepackage{multirow}
% koma options
\addtokomafont{section}{\sffamily} %Kapitel
\addtokomafont{sectioning}{\sffamily} %Titelzeilen
\deffootnote{1em}{1em}{\textsuperscript{\thefootnotemark\ }}
\setcounter{tocdepth}{4}
\setcounter{secnumdepth}{4}
%\usepackage[subfigure]{tocloft}
\usepackage[subfigure]{tocloft}
\captionsetup[subfigure]{position=top}
\setlength{\cftfignumwidth}{3em}
\setlength{\cfttabnumwidth}{3em}
\usepackage[hidelinks]{hyperref}

\begin{document}

% Pagination
\clearpage
\setstretch{1.25}
\pagestyle{plain}
%\cfoot{\thepage}
\pagenumbering{arabic}
\setcounter{page}{1}

\begin{titlepage}
	\selectlanguage{english}

	\title{\scshape{Reproduction of Paper Results}}
	\subtitle{\sffamily{Gender identity and wives' labor market outcomes in West and East Germany between 1983 and 2016}}
	\author{Maximilian Sprengholz\thanks{Humboldt-Universiät zu Berlin, Institut für Sozialwissenschaften; DIW Berlin until 31st January 2019; \href{mailto:maximilian.sprengholz@hu-berlin.de}{maximilian.sprengholz@hu-berlin.de}} \and Anna Wieber\thanks{Deutsche Rentenversicherung Bund; DIW Berlin until 31st December 2014; \href{mailto:Anna.Wieber@drv-bund.de}{Anna.Wieber@drv-bund.de}} \and Elke Holst\thanks{DIW Berlin; \href{mailto:eholst@diw.de}{eholst@diw.de}}}

	\date{\today}

	\maketitle

	\centering
	Supplementary material to: \href{https://doi.org/10.1093/ser/mwaa048}{https://doi.org/10.1093/ser/mwaa048}.
\end{titlepage}

%
% Summary statistics
%

\clearpage
\begin{landscape}
	\section{Panel summary statistics}\label{sumstats}
	\setcounter{table}{0}
	\setcounter{figure}{0}

	\begin{table}[H]
		\begin{center}
			\begin{scriptsize}
				\vspace*{3mm}
				\captionof{table}{Panel summary statistics}
				\sisetup{
					table-text-alignment=right,
					table-number-alignment=right,
					table-unit-alignment=right,
					table-figures-integer = 1,
					table-figures-decimal=2,
					input-decimal-markers =	.	,
					input-symbols = , ,
					table-align-text-post = false,
					table-column-width = 1.2cm,
					%table-space-text-post = \sym{***},
					%table-space-text-post = \sym{\dag},
					%table-space-text-pre = {(},
					%table-space-text-post = {)}
				}
				\noindent
				\strutlongstacks{T}
				{\renewcommand{\arraystretch}{1.0}
					\begin{tabularx} {23.7cm} {@{} X
							S S S S S S S S S S S S S S
							@{}}
						\hline\noalign{\smallskip}\noalign{\smallskip}
						& {Mean} & {Std. Dev.} & {Mean} & {Std. Dev.} & {Mean} & {Std. Dev.} & {Mean} & {Std. Dev.} & {Mean} & {Std. Dev.}\\
\noalign{\smallskip}\hline \noalign{\smallskip} \noalign{\smallskip}\textbf{Working couples}\\ \noalign{\smallskip}\hline \noalign{\smallskip}\emph{Husband}\\ \noalign{\smallskip}{Annual labor income (gross)} & 25362.91 & 12540.10 & 37564.34 & 20832.64 & 45970.85 & 26934.87 & 23894.46 & 11981.66 & 33520.57 & 18484.68\\
  \noalign{\smallskip}{Full-time share of worked months} & 0.99 & 0.07 & 0.98 & 0.12 & 0.96 & 0.19 & 0.98 & 0.12 & 0.97 & 0.16\\
 \noalign{\smallskip}\hline\noalign{\smallskip}\emph{Wife}\\ \noalign{\smallskip}{Annual labor income (gross)} & 11213.17 & 9528.46 & 16416.17 & 13063.14 & 19542.94 & 16522.70 & 18921.02 & 9624.75 & 24550.28 & 13829.42\\
  \noalign{\smallskip}{Full-time share of worked months} & 0.42 & 0.49 & 0.37 & 0.48 & 0.29 & 0.45 & 0.73 & 0.44 & 0.56 & 0.49\\
  \noalign{\smallskip}{Wife's share of household income (gross)} & 0.29 & 0.16 & 0.29 & 0.17 & 0.29 & 0.17 & 0.44 & 0.13 & 0.42 & 0.16\\
  \noalign{\smallskip}{Wife earns more} & 0.08 & 0.27 & 0.10 & 0.30 & 0.11 & 0.31 & 0.34 & 0.47 & 0.33 & 0.47\\
\noalign{\smallskip}\hline\noalign{\smallskip}{N} & 7194 &  & 14368 &  & 19898 &  & 5097 &  & 4788 & \\
{n} & 1984 &  & 3796 &  & 5485 &  & 1196 &  & 1123 & \\
 \noalign{\smallskip}\hline\noalign{\smallskip}\noalign{\smallskip}\textbf{All couples}\\ \noalign{\smallskip}\hline \noalign{\smallskip}\emph{Husband}\\  \noalign{\smallskip}{Employed} & 0.89 & 0.32 & 0.86 & 0.35 & 0.87 & 0.34 & 0.80 & 0.40 & 0.82 & 0.38\\
 \noalign{\smallskip}\hline\noalign{\smallskip}\emph{Wife}\\ \noalign{\smallskip}{Employed} & 0.52 & 0.50 & 0.61 & 0.49 & 0.73 & 0.45 & 0.69 & 0.46 & 0.77 & 0.42\\
\noalign{\smallskip}\hline\noalign{\smallskip}{N} & 21140 &  & 39759 &  & 48585 &  & 14103 &  & 11976 & \\
{n} & 3982 &  & 7459 &  & 11164 &  & 2197 &  & 2389 & \\

						\noalign{\smallskip}\hline
					\end{tabularx}
				}
				\captionsetup{
					justification=justified,
					textfont={footnotesize, normalfont},
				}
				\caption*{\scriptsize{Note: Weighted. `Working couples' refers to the sample restrictions outlined in this section. `All couples' refers to all married couples between age 25 and 64 in the data. Income related statistics are based on (once) de-rounded income values. Income reported in Euro. As the annual income is calculated based on answers by survey respondents in the subsequent year, income related statistics do not include values for 2016.}}
			\end{scriptsize}
			\normalsize
		\end{center}
	\end{table}
\end{landscape}
\clearpage

\captionsetup[table]{
	belowskip=-7pt
}

%
% DCD
%

\section{Discontinuity estimates: Main model}

% Table command
\newcommand{\dcdtable}[3]{
	\begin{table}[H]
		\begin{center}
			\begin{scriptsize}
				\vspace*{3mm}
				\captionof{table}{#1}
				\sisetup{
					table-column-width = 0.6cm,
					table-text-alignment=center,
					table-number-alignment=center,
					table-unit-alignment=center,
					input-ignore={,},
					input-decimal-markers={. ,},
					input-symbols = . ,
					group-separator={,},
					group-minimum-digits=3,
					table-align-text-post = false
				}
				\noindent
				\strutlongstacks{T}
				{\renewcommand{\arraystretch}{1.3}
					\begin{tabularx} {\textwidth} {@{} X
							S[table-column-width = 1.1cm, table-format=1.2]
							S[table-column-width = 0.7cm, table-format=1.2]
							S[table-column-width = 0.8cm, table-format=2.2]
							S[table-column-width = 0.8cm, table-format=4.0, round-mode=places,round-precision=0]
							S[table-column-width = 0.9cm, table-format=1.2]
							S[table-column-width = 0.6cm, table-format=1.2]
							S[table-column-width = 0.8cm, table-format=2.2]
							S[table-column-width = 0.8cm, table-format=4.0, round-mode=places,round-precision=0 ]
							S[table-column-width = 0.9cm, table-format=1.2]
							S[table-column-width = 0.7cm, table-format=1.2]
							@{}}
						\midrule
						\input{../tables/#2.tex}
						\midrule
					\end{tabularx}
				}
				\captionsetup{
					justification=justified,
					textfont={footnotesize, normalfont},
				}
				\caption*{\scriptsize{\dcdnoteA\,#3\,\dcdnoteB}}
			\end{scriptsize}
			\normalsize
		\end{center}
	\end{table}
}

% Note commands
\newcommand{\dcdnoteA}{Note: Discontinuity estimated for the distribution of the wife's share of household income using a McCrary Test \citeyear{mccrary2008manipulation}. Restricted to married couples where spouses are between age 25 and 64, where wife and husband have positive annual income, worked at least one month, and are not in education, vocational training, civil service or parental leave, are not self-employed, and do not receive unemployment benefits or pensions in year t$-1$.}

\newcommand{\samplenote}{}

\newcommand{\specnote}{Cut-off at 0.50001, bin size 0.05, optimal (automatic) bandwidth.}

\newcommand{\dcdnoteB}{Averaged over 100 simulation runs based on procedures described in the data section. As annual income is calculated based on answers by survey respondents in the subsequent year, income related statistics for the period 2004-2016 are based on annual incomes between 2004 and 2015. Significance of differences based on two-tailed t-test; *** p$<$0.001, ** p$<$0.01, * p$<$0.05, \dag\, p$<$0.1}

\renewcommand{\samplenote}{}

\dcdtable{Density discontinuity estimates}%
{dcd_wis_pc1_random_codot50001_bs05_all_dspike}%
{\samplenote\,\specnote}

%
% FE
%

% Robustness II
\section{Panel regresssion: Main model}

% Table command
\newcommand{\xtfetable}[3]{
	\begin{table}[H]
		\begin{center}
			\begin{scriptsize}
				\vspace*{3mm}
				\captionof{table}{#1}
				\sisetup{
					table-column-width = 1.75cm,
					table-text-alignment=center,
					table-number-alignment=center,
					table-unit-alignment=center,
					table-figures-integer = 1,
					table-figures-decimal=3,
					input-decimal-markers =	.	,
					input-symbols = , ,
					table-align-text-post = false,
					group-separator={,},
					group-minimum-digits={3},
					output-decimal-marker={.},
					table-space-text-post = \sym{***},
					table-space-text-post = \sym{\dag},
					table-space-text-pre = {(},
					table-space-text-post = {)}
				}
				\noindent
				\strutlongstacks{T}
				{\renewcommand{\arraystretch}{1.3}
					\begin{tabularx} {\textwidth} {@{} X S[table-align-text-post=false] S S S S S S @{}}
						\midrule
						\input{../tables/#2.tex}
						\midrule
					\end{tabularx}
				}
				\captionsetup{
					justification=justified,
					textfont={footnotesize, normalfont},
				}
				\caption*{\scriptsize{\xtfenoteA\,#3\,\xtfenoteB}}
			\end{scriptsize}
			\normalsize
		\end{center}
	\end{table}
}

% Note commands
\newcommand{\xtfenoteA}{Notes: Unweighted regressions with couple fixed effects. $wifeEarnsMore$ is a dummy denoting if the wife earned more than her husband in a given year. Restricted to married couples where spouses are between age 25 and 64, where wife and husband have positive annual income, worked at least one month, and are not in education, vocational training, civil service or parental leave, are not self-employed, and do not receive unemployment benefits or pensions in year t$-1$.}

\renewcommand{\samplenote}{}

\renewcommand{\specnote}{}

\newcommand{\xtfenoteB}{All models include dummies for age groups of 5 years for both spouses, dummy variables for the age of the youngest child in the household, the district level unemployment rate at the month of the interview, and dummy variables for the survey year -- all at t. Averaged over 25 simulation runs based on procedures described in the data section. Standard errors clustered on couple level in parentheses; *** p$<$0.001, ** p$<$0.01, * p$<$0.05, \dag\, p$<$0.1}

\renewcommand{\specnote}{All models include a cubic polynomial of the logarithm of the wife's and the husband's annual income, an interaction between the wife's and the husband's logarithmic annual income, and dummies indicating if the income values for wife or husband have been imputed -- all at t$-1$.}

% Full sample, cubic income spec

% LFP
\xtfetable{Wife is employed in t}%
{f1yw_pc1_cciaimp_all_dspike}%
{\samplenote\,\specnote}

% FTPT
\xtfetable{Full-time share of wife's worked months in t}%
{f1ftpt_pc1_cciaimp_all_dspike}%
{\samplenote\,\specnote}

% VEBZT
\xtfetable{Wife's weekly working hours in t at time of interview}%
{vebzt_pc1_cciaimp_all_dspike}%
{\samplenote\,\specnote}

%
% References
%

\pagestyle{plain}
\bibliographystyle{apalike}
\bibliography{gi}
\clearpage

\end{document}
